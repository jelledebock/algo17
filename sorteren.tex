\documentclass[a4paper]{article}
\title{Samenvatting algoritmen}
\author{Aaron Mousavi (aanvullingen Jelle De Bock)}
\date{\today}
\usepackage{fullpage}
\usepackage[T1]{fontenc}
\usepackage[utf8]{inputenc}
\usepackage{hyperref}
\usepackage{graphicx}
\usepackage{listings}
\usepackage[usenames,dvipsnames,svgnames,table]{xcolor}

\definecolor{listinggray}{gray}{0.9}
\definecolor{lbcolor}{rgb}{0.9,0.9,0.9}

\lstset{%
	backgroundcolor=\color{lbcolor},
	tabsize=4,
	language=C++,
	captionpos=b,
	frame=lines,
	numbers=left,
	numbersep=5pt,
	breaklines=true,
	keywordstyle=\color[rgb]{0,0,1},
	commentstyle=\color{DarkGreen},
	stringstyle=\color{red}
}

\begin{document}

\maketitle
\newpage
\section{Asymptotische notaties}
De naam ``asymptotische notatie'' is gekozen omdat deze notaties enkel geldig zijn voor grote waarden van $n$.

\begin{itemize}
	\item $\Theta(n)$: de functie is langs boven en onder begrensd met de constanten $c_1$ en $c_2$, daarom ligt de uitvoeringstijd tussen $c_1 \cdot f(n)$ en $c_2 \cdot f(n)$ ligt.
	\item $O(n)$: is maximum uitvoeringstijd, het zal nooit slechter zijn dat dit. Er wordt geprobeerd om de laagste, bewezen waarde op te schrijven. We kunnen ook zeggen dat het worst case scenario $\Theta(n)$ is, maar de big O is een sterkere notatie.
	\item $\Omega(n)$: de ondergrens, het algoritme heeft een uitvoeringstijd van minstens $n$. 
\end{itemize}
\newpage

\section{Selection sort}
\subsection*{Kenmerken}
\begin{itemize}
	\item Niet stabiel
	\item Rangschikt ter plaatste
	\item $\Theta(n^2)$
	\item Ofwel het kleinste zoeken en vooraan zetten, of het grootste en achteraan zetten.
\end{itemize}

\subsection*{Algoritme}
\begin{lstlisting}[language=C++]
template <typename T>
void SelectionSort<T>::operator()(vector<T> &vec) const {
	for(int i=vec.size()-1; i>=0; i--){
		int maxIndex=zoekGrootste(vec, i);
		swap(vec[i], vec[maxIndex]);
    }
}

template <typename T>
int SelectionSort<T>::zoekGrootste(const vector<T> & vec, int eindIndex) const {
	int maxIndex = 0;
	for(int i=1; i<=eindIndex; i++){
		if(vec[i] > vec[maxIndex]){
			maxIndex = i;
		}
	}
	return maxIndex;
}
\end{lstlisting}

\subsection*{Opmerkingen}
\begin{itemize}
	\item Lidfuncties zijn const, en de vector in de zoekGrootste ook!
\end{itemize}
\newpage

\section{Insertion sort}
\subsection*{Kenmerken}
\begin{itemize}
	\item Stabiel
	\item Rangschikt ter plaatste
	\item $O(n^2)$
	\item $\Omega(n)$
\end{itemize}

\subsection*{Algoritme}
\begin{lstlisting}[language=C++]
template <typename T>
void InsertionSort<T>::operator()(vector<T> &vec) const {
	for(int i=1; i<vec.size(); i++){
		 T help;
        //Voeg element i toe aan het gesorteerde deel 
        //(de elementen voor i staan in volgorde)
        help = std::move(v[i]);
        int j = i-1;
        while(j>=0 && help<v[j]){
            //help is kleiner dan het voorgaande element dus
            //laat element op j plaats maken (schuif op)
            v[j+1] = std::move(v[j]);
            j--;
        }
        //na de voorgaande lus hebben we de positie van help gevonden
        //merk op: j+1 we zijn namelijk 1tje te ver gegaan in de while lus
        v[j+1] = std::move(help);
	}
}
\end{lstlisting}

\subsection*{Opmerkingen}
\begin{itemize}
	\item Const niet vergeten.
	\item Moves!
	\item i begint bij 1, regel 8 zeker niet vergeten en op het laatste is het ook j+1!
\end{itemize}
\newpage

\section{Shell sort}
\subsection*{Kenmerken}
\begin{itemize}
	\item Niet stabiel
	\item Rangschikt ter plaatste
	\item $O(n^2)$
	\item $\Omega(n \cdot log(n))$
\end{itemize}

\subsection*{Algoritme}
\begin{lstlisting}[language=C++]
//Sedgewick's increments, used in the shellsort algorithm
vector<int> INCREMENTS = {  1391376, 463792, 198768, 86961, 33936,
                            13776, 4592, 1968, 861, 336,
                            112, 48, 21, 7, 3, 1
};

template <typename T>
void Shellsort<T>::operator()(vector<T> & v) const {
	//Ga over alle increments
    for(int i = 0 ; i < INCREMENTS.size() ; i++){
        int increment = INCREMENTS[i];
        //Controleer of increment niet te groot is voor onze array
        if(increment < v.size()){
            //Ga k-sorteren voor alle elementen startende van het increment
            for(int j = increment ; j < v.size() ; j++){
                T temp = v[j];
                //Insertion sort stap, plaats j-de element op zijn juiste plek, telkens increment-stapjes verder springend
                int k = j-increment;
                while(k>=0 && temp < v[k]){
                	v[k+increment] = v[k];
                	k-=increment;
                }
                v[k+increment]=std::move(temp);
                
                /* Evenwaardig getest alternatief 
		            int k = j;
		            
		            while(k-increment>=0 && v[k-increment]>temp){
		                v[k]=std::move(v[k-increment]);
		                k-=increment;
		            }
		            v[k]=std::move(temp);
                */
            }
        }
    }
}
\end{lstlisting}

\subsection*{Opmerkingen}
\begin{itemize}
	\item Bij de laatste regel weer de +incr niet vergeten!
\end{itemize}
\newpage

\section{Heap sort}
\subsection*{Kernmerken}
\begin{itemize}
	\item Niet Stabiel
	\item Ter plaatse
	\item $\Theta(n \cdot log(n))$
\end{itemize}

\subsection*{Algoritme heap sort}
\begin{lstlisting}
template <typename T>
void HeapSort<T>::operator()(vector<T> &vec) const {
	Heap<T> heap(vec, vec.size());
	heap.buildHeap();
	for (size_t i = vec.size()-1; i > 0; i--) { // opgepast i > 0!!!
		// zet wortel (= grootste achteraan) en zet leaf bovenaan
		T wortel = move(*heap.geefWortel());
		heap.vervangWortel(vec[i]);
		vec[i] = move(wortel);
		heap.percolateDown(0, i);
	}
}
\end{lstlisting}

\subsection*{Algoritmes heap}
\begin{lstlisting}
template<class T>
void Heap<T>::buildHeap(){
	// dit moet enkel voor de bovenste helft
	for (int i = vec.size()/2; i >= 0; i--) {
		percolateDown(i, vec.size());
	}
}

template<class T>
const T* Heap<T>::geefWortel() const {
	return &vec[0];
}

template<class T>
int Heap<T>::getGrootsteKind(int parentIndex, int length){
	int kind = parentIndex * 2 + 1; // moet bestaan, dus niet oproepen op leaves! Plus 1 want zero based.
	if(kind + 1 < length){
		if(vec[kind+1] > vec[kind]){
			kind+=1;
		}
	}
	return kind;
}

template<class T>
void Heap<T>::percolateDown(size_t parentIndex, size_t length){
	if(parentIndex * 2 + 1 >= length) {
		return; // parent heeft geen kinderen, en kan dus ook niet zakken
	}

	T parent = move(vec[parentIndex]);
	size_t grootsteKindIndex = getGrootsteKind(parentIndex, length);
	// blijf dit doen tot er geen (groter) kind meer is
	while ((2*parentIndex+1<length) && vec[grootsteKindIndex] > parent) {
		// parent wordt child
		vec[parentIndex] = vec[grootsteKindIndex];
		parentIndex = grootsteKindIndex;
	}
	// alle elementen die groter waren dan de originele parent staan er nu boven, zet nu de oude parent op de laatste plaats
	// die waarde hier zit toch al in het niveau erboven
	vec[parentIndex] = move(parent);
}

template<class T>
void Heap<T>::vervangWortel(const T& ele){
	vec[0] = move(ele);
}
\end{lstlisting}

\subsection*{Opmerkingen}
\newpage

\section{Merge sort}
\subsection*{Kernmerken}
\begin{itemize}
	\item Stabiel (deze versie toch)
	\item Niet ter plaatse
	\item $\Theta(n \cdot log(n))$
\end{itemize}

\subsection*{Top down Algoritme}
\begin{lstlisting}
template <typename T>
void Mergesort<T>::mergesort(vector<T> &h, vector<T> &v, int l, int r) const{
    if(l+1 < r){
        int m = l+(r-l)/2;
        mergesort(v, h, l, m);
        mergesort(v, h, m, r);

        merge(h, v, l, m , r);
    }
}

template <typename T>
void Mergesort<T>::operator()(vector <T> &v) const {
    vector<T> other(v);  //de andere tabel, waarin we telkens de rechter deeltabel zullen in opslaan
    mergesort(other,v,0,v.size());
}

template <typename T>
void merge(vector<T> &from, vector<T> &to, int l, int m, int r){
    int i=l;
    int j=m;

    //je weet exact hoeveel keer je een element gaat plaatsen (dus tellertje k)
    for(int k=i; k<r; k++){
        if(i<m && (j>=r || from[i] <= from[j])){
            to[k]=from[i++];
        }else{
            to[k]=from[j++];
        }
    }
}
\end{lstlisting}

\subsection*{Bottom up Algoritme}
\begin{lstlisting}
template <typename T>
void Mergesort_bu<T>::operator()(vector <T> &v) const {
    vector<T> temp(v.size());
    int breedte=1;
    while(breedte<v.size()){
        int l=0;
        while(l+breedte<v.size()){
            merge_bu(v, temp, l, breedte);
            l=l+breedte*2;
        }
        breedte*=2;
    }
}

template <typename T>
void merge_bu(vector<T> &vec, vector<T> &temp, int l, int width){
    int i=l;
    int j=l+width;

    int m = l+width;
    int r = l+2*width>vec.size()?vec.size():l+2*width;
    int k=l;

    for(int s=l; s<r; s++){
        if(i<m && (j>=r || vec[i] <= vec[j])){
            temp[s]=vec[i++];
        }else{
            temp[s]=vec[j++];
        }
    }

    for(int s=l; s<r; s++){
        vec[s]=temp[s];
    }
}
\end{lstlisting}
\subsection*{Opmerkingen}
\begin{itemize}
	\item Hoe doe je dit met een hulp array van $n/2$?
\end{itemize}
\newpage

\section{Quick sort}
\subsection*{Single pivot}
\begin{lstlisting}
template <typename T>
void Quicksort<T>::operator()(vector <T> &v) const {
    quicksort(v,0, v.size()-1);
}

template <typename T>
void Quicksort<T>::quicksort(vector <T> &v, int l, int r) const {
    //Rangschik deelvector v[l]..[r]
    if(l<r) {
        T pivot = v[l];
        int i=l, j=r;
        while(v[j]>pivot){
            j--;
        }
        while(i<j){
            swap(v[i],v[j]);
            i++;
            while(v[i]<pivot)
                i++;
            j--;
            while(v[j]>pivot)
                j--;
        }
        quicksort(v, l, j);
        quicksort(v, j+1, r);
    }
}
\end{lstlisting}

\subsection*{Dual Pivot (JDK aka Yaroslavskiy)}
\begin{lstlisting}
template <typename T>
void Quicksort_Dual_Pivot<T>::operator()(vector <T> &v) const {
    quicksort_Dual_Pivot(v,0, v.size()-1);
}


template <typename T>
void Quicksort_Dual_Pivot<T>::quicksort_Dual_Pivot(vector <T> &v, int l, int r) const {
    if (l < r) {
        //Pivots
        T p1 = v[l];
        T p2 = v[r];

        //Swap the 2 pivots if p1>p2
        if (p1 > p2) {
            v[l] = std::move(p2);
            v[r] = std::move(p1);
            p1 = v[l];
            p2 = v[r];
        }
		//m is het tellertje waarmee we door de tabel lopen
		//k is de grens tussen linker en middelste gedeelte (alle waardes <p1)
		//g is de grens tussen rechter en middelste gedeelte (alle waardes >p2)
        int k = l + 1;
        int g = r - 1;
        int m = k;

        while (m <= g) {
            if (v[m] < p1) {
                T temp = std::move(v[k]);
                v[k] = std::move(v[m]);
                v[m] = std::move(temp);
                k++;
            } else if (v[m] >= p2) {
                while (v[g] > p2 && m < g) {
                    g--;
                }
                T temp = std::move(v[g]);
                v[g] = std::move(v[m]);
                v[m] = std::move(temp);

                g--;

                if (v[m] < p1) {
                    T temp = std::move(v[m]);
                    v[m] = std::move(v[k]);
                    v[k] = temp;
                    k++;
                }
            }
            m++;
        }
        //Zet pivots op hun plek
        k = k - 1;
        g = g + 1;
        T temp = std::move(v[k]);
        v[k] = std::move(v[l]);
        v[l] = std::move(temp);

        temp = std::move(v[g]);
        v[g] = std::move(v[r]);
        v[r] = std::move(temp);

        quicksort_Dual_Pivot(v, l, k - 1);
        quicksort_Dual_Pivot(v, k + 1, g - 1);
        quicksort_Dual_Pivot(v, g + 1, r);
    }
}
\end{lstlisting}
\newpage


\newpage

\section{Counting sort}
\subsection*{Kenmerken}
\begin{itemize}
	\item Stabiel
	\item Niet ter plaatste
	\item Sleutels moeten gehele getallen zijn
	\item Beperkt interval
	\item $\Theta(n+k)$ met $n$ aantal elementen en $k$ het aantal verschillende sleutels.
\end{itemize}

\subsection*{Algoritme}
\begin{lstlisting}
template <typename T>
void CountingSort<T>::operator()(vector<T> &vec) const {

	// frequentietabel opstellen
	vector<int> freq(vec.size(), 0);;
	for (int i = 0; i < vec.size(); i++) {
		freq[vec[i]]++; // ervan uitgaande dat vec een getal teruggeeft...
	}

	// cumulatieve tabel maken
	for (int i = 1; i < freq.size(); i++) {
		freq[i] += freq[i-1];
	}
	
	// output tabel maken met het resultaat
	// overloop terug de originele tabel
	vector<int> output(vec.size());
	for (int i = 0; i < vec.size(); i++) {
		output[freq[vec[i]]-1] = move(vec[i]); // -1!!
		freq[vec[i]]--;
	}

	// output kopieren naar de originele tabel
	for (int i = 0; i < vec.size(); i++) {
		vec[i] = move(output[i]);
	}
}
\end{lstlisting}

\subsection*{Opmerkingen}
\newpage

\section{Radix sort}
\begin{lstlisting}
// A utility function to get maximum value in arr[]
int getMax(int arr[], int n)
{
    int mx = arr[0];
    for (int i = 1; i < n; i++)
        if (arr[i] > mx)
            mx = arr[i];
    return mx;
}
 
// A function to do counting sort of arr[] according to
// the digit represented by exp.
void countSort(int arr[], int n, int exp)
{
    int output[n]; // output array
    int i, count[10] = {0};
 
    // Store count of occurrences in count[]
    for (i = 0; i < n; i++)
        count[ (arr[i]/exp)%10 ]++;
 
    // Change count[i] so that count[i] now contains actual
    //  position of this digit in output[]
    for (i = 1; i < 10; i++)
        count[i] += count[i - 1];
 
    // Build the output array
    for (i = n - 1; i >= 0; i--)
    {
        output[count[ (arr[i]/exp)%10 ] - 1] = arr[i];
        count[ (arr[i]/exp)%10 ]--;
    }
 
    // Copy the output array to arr[], so that arr[] now
    // contains sorted numbers according to current digit
    for (i = 0; i < n; i++)
        arr[i] = output[i];
}
 
// The main function to that sorts arr[] of size n using 
// Radix Sort
void radixsort(int arr[], int n)
{
    // Find the maximum number to know number of digits
    int m = getMax(arr, n);
 
    // Do counting sort for every digit. Note that instead
    // of passing digit number, exp is passed. exp is 10^i
    // where i is current digit number
    for (int exp = 1; m/exp > 0; exp *= 10)
        countSort(arr, n, exp);
}
\end{lstlisting}
\newpage

\section{Bucket sort}
\subsection*{Kenmerken}
\begin{itemize}
	\item Stabiel: door insertion sort
	\item Niet ter plaatste
	\item $O(n^2)$: alles in \'{e}\'{e}n bucket en insertionsort
	\item $\Theta(n)$
\end{itemize}

\subsection*{Algoritme}

\begin{lstlisting}
template <typename T>
T BucketSort<T>::grootsteValue(vector<T> const &vec) const {
	// T moet wel zeker getal zijn, of anders een complexere interface maken
	T grootste = vec[0];
	for (int i = 0; i < vec.size(); i++) {
		if(vec[i] > grootste) grootste = vec[i];
	}
	return grootste;
}

template <typename T>
void BucketSort<T>::operator()(vector<T> &vec) const {
	// dit houdt alle buckets bij
	vector<vector<T>> buckets(vec.size());
	T grootste = grootsteValue(vec);

	// overloop alle waarden
	for (int i = 0; i < vec.size(); i++) {
		// voorbeeld hash functie: (value*size)/(max value+1)
		int bucket = (vec[i] * vec.size())/(grootste+1);
		buckets[bucket].push_back(move(vec[i]));
	}
	
	// sorteer alle buckets
	InsertionSort<T> insertionSort;
	for (int i = 0; i < buckets.size(); i++) {
		insertionSort(buckets[i]);
	}

	// leeg alle buckets terug in de array
	int index=0;
	for (int i = 0; i < buckets.size(); i++) {
		for (int j = 0; j < buckets[i].size(); j++) {
			vec[index++] = move(buckets[i][j]);
		}
	}
}
\end{lstlisting}

\subsection*{Opmerkingen}
\newpage

\section{Mogelijks handig}
\subsection*{Move implementatie voorbeeld}
\label{sub:Move implementatie voorbeeld}

\begin{lstlisting}
// Move constructor.
MemoryBlock(MemoryBlock&& other)
   : _data(nullptr)
   , _length(0)
{
   std::cout << "In MemoryBlock(MemoryBlock&&). length = " 
             << other._length << ". Moving resource." << std::endl;

   // Copy the data pointer and its length from the 
   // source object.
   _data = other._data;
   _length = other._length;

   // Release the data pointer from the source object so that
   // the destructor does not free the memory multiple times.
   other._data = nullptr;
   other._length = 0;
}

// Move assignment operator.
MemoryBlock& operator=(MemoryBlock&& other)
{
   std::cout << "In operator=(MemoryBlock&&). length = " 
             << other._length << "." << std::endl;

   if (this != &other)
   {
      // Free the existing resource.
      delete[] _data;

      // Copy the data pointer and its length from the 
      // source object.
      _data = other._data;
      _length = other._length;

      // Release the data pointer from the source object so that
      // the destructor does not free the memory multiple times.
      other._data = nullptr;
      other._length = 0;
   }
   return *this;
}
\end{lstlisting}
Source: \url{https://msdn.microsoft.com/en-us/library/dd293665.aspx}.
\end{document}
